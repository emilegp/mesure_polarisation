\documentclass[11pt,letterpaper]{article}
\usepackage[top=3cm, bottom=2cm, left=2cm, right=2cm, columnsep=20pt]{geometry}
\usepackage{pdfpages}
\usepackage{graphicx}
\usepackage{etoolbox}
\apptocmd{\sloppy}{\hbadness 10000\relax}{}{}
% \usepackage[numbers]{natbib}
\usepackage[T1]{fontenc}
\usepackage{ragged2e}
\usepackage[french]{babel}
\usepackage{listings}
\usepackage{color}
\usepackage{soul}
\usepackage[utf8]{inputenc}
\usepackage[export]{adjustbox}
\usepackage{caption}
\usepackage{amsmath}
\usepackage{amssymb}
\usepackage{float}
\usepackage{csquotes}
\usepackage{fancyhdr}
\usepackage{wallpaper}
\usepackage{siunitx}
\usepackage[indent]{parskip}
\usepackage{textcomp}
\usepackage{gensymb}
\usepackage{multirow}
\usepackage[hidelinks]{hyperref}
\usepackage{abstract}
\usepackage{braket}
\renewcommand{\abstractnamefont}{\normalfont\bfseries}
\renewcommand{\abstracttextfont}{\normalfont\itshape}
\usepackage{titlesec}
\titleformat{\section}{\large\bfseries}{\thesection}{1em}{}
\titleformat{\subsection}{\normalsize\bfseries}{\thesubsection}{1em}{}
\titleformat{\subsubsection}{\normalsize\bfseries}{\thesubsubsection}{1em}{}

\usepackage{xcolor}
\definecolor{codegreen}{rgb}{0,0.6,0}
\definecolor{codegray}{rgb}{0.5,0.5,0.5}
\definecolor{codepurple}{rgb}{0.58,0,0.82}
\definecolor{backcolour}{rgb}{0.95,0.95,0.92}
\lstdefinestyle{mystyle}{
    backgroundcolor=\color{backcolour},   
    commentstyle=\color{codegreen},
    keywordstyle=\color{magenta},
    numberstyle=\tiny\color{codegray},
    stringstyle=\color{codepurple},
    basicstyle=\ttfamily\footnotesize,
    breakatwhitespace=false,         
    breaklines=true,                 
    captionpos=b,                    
    keepspaces=true,                 
    numbers=left,                    
    numbersep=5pt,                  
    showspaces=false,                
    showstringspaces=false,
    showtabs=false,                  
    tabsize=2
}
\lstset{style=mystyle}

\usepackage[most]{tcolorbox}
\newtcolorbox{note}[1][]{
  enhanced jigsaw,
  borderline west={2pt}{0pt}{black},
  sharp corners,
  boxrule=0pt, 
  fonttitle={\large\bfseries},
  coltitle={black},
  title={Note:\ },
  attach title to upper,
  #1
}

%----------------------------------------------------

\newcommand*{\Coord}[2]{% 
    \begin{pmatrix} 
      #1\\ 
      #2 
    \end{pmatrix}}

\setlength{\parindent}{0pt}
\DeclareCaptionLabelFormat{mycaptionlabel}{#1 #2}
\captionsetup[figure]{labelsep=colon}
\captionsetup{labelformat=mycaptionlabel}
\captionsetup[figure]{name={Figure }}
\newcommand{\inlinecode}{\normalfont\texttt}
\usepackage{enumitem}
\setlist[itemize]{label=\textbullet}

\begin{document}
\begin{titlepage}
\center

\begin{figure}
    \ThisULCornerWallPaper{.4}{Polytechnique_signature-RGB-gauche_FR.png}
\end{figure}
\vspace*{2 cm}

\textsc{\Large \textbf{PHS2223 --} Introduction à l'optique moderne}\\[0.5cm]
\large{\textbf{Équipe : 04}}\\[1.5cm]

\rule{\linewidth}{0.5mm} \\[0.5cm]
\Large{\textbf{Expérience 3}} \\[0.2cm]
\text{Mesure de polarisation}\\
\rule{\linewidth}{0.2mm} \\[2.3cm]

\large{\textbf{Présenté à}\\
  Guillaume Sheehy\\
  Esmat Zamani\\[2.5cm]
  \textbf{Par :}\\
  Émile \textbf{Guertin-Picard} (2208363)\\
  Laura-Li \textbf{Gilbert} (2204234)\\
  Tom \textbf{Dessauvages} (2133573)\\[3cm]}

\large{\today\\
Département de Génie Physique\\
Polytechnique Montréal\\}

\end{titlepage}

%----------------------------------------------------

\tableofcontents
\pagenumbering{roman}
\newpage

\pagestyle{fancy}
\setlength{\headheight}{14pt}
\renewcommand{\headrulewidth}{0pt}
\fancyfoot[R]{\thepage}

\pagestyle{fancy}
\fancyhf{}
\renewcommand{\headrulewidth}{1pt}
\fancyhead[L]{\textbf{PHS2223}}
\fancyhead[C]{Mesure de polarisation}
\fancyhead[R]{\today}
\fancyfoot[R]{\thepage}

\pagenumbering{arabic}
\setcounter{page}{1}

%----------------------------------------------------

\section{Résultats}

Suite à la prise de données au laboratoire, il est possible de comparer les valeurs de
coefficients de transmission obtenus aux hypothèses émises pour les mesures à deux et trois polariseurs.
En reprenant la courbe proportionnelle à $\cos^{2}\left( \theta \right)$ pour le montage de
deux polariseurs, la figure \ref{2pol} montre les valeurs obtenues, qui suivent très bien la
tendance prédite.

\begin{figure}[H]
  \centering
  \includegraphics[scale=0.7]{viz_deux_pol.png}
  \caption{Résultats de la prise de mesure superposée à la prédiction de l'hypothèse pour la 
  mesure à deux polariseurs.}
  \label{2pol}
\end{figure}

De même pour le montage à trois polariseurs, la figure \ref{3pol} montre les valeurs de coefficient
de transmission expérimentales, qui suivent aussi bien la courbe proportionnelle à 
$\cos^{4}\left( \theta \right)$ émise par l'hypothèse.

\begin{figure}[H]
  \centering
  \includegraphics[scale=0.7]{viz_trois_pol.png}
  \caption{Résultats de la prise de mesure superposée à la prédiction de l'hypothèse pour la 
  mesure à trois polariseurs.}
  \label{3pol}
\end{figure}

\subsection{Estimation des erreurs}

Les incertitudes présentées avec les barres d'erreurs dans les graphiques ont été trouvées comme suit.
Pour les incertitudes sur l'angle $\theta$, la valeur est tout simplement la moitié de la plus petite
graduation sur les cadrans gradués, multiplié par le nombre de cadrans utilisés. Ainsi, pour les
mesures à deux polariseurs, comme la plus petite graduation était de $1\degree$, l'incertitude 
$\Delta\theta$  sur l'angle est aussi de $1\degree$. Pour trois polariseurs, $\Delta\theta = 1.5\degree$. 

Pour les incertitudes sur les coefficients de transmission $C$, dont le calcul se fait par

\begin{equation}
  C = \frac{P}{P_{0}},
\end{equation}

où $P$ est la puissance mesurée à un certain angle $\theta$ et $P_{0}$ est $P\left( \theta= 0 \right)$.
La propagation d'erreur pour une division de la sorte permet de calculer l'incertitude sur $C$ pour un
certain $P$ \textcolor{red}{source A} :

\begin{equation}
  \Delta C = C\sqrt{\left( \frac{\Delta P}{P} \right)^{2} + \left( \frac{\Delta P_{0}}{P_{0}} \right)^{2}},
\end{equation}

où $\Delta P$ et $\Delta P_{0}$ sont des incertitudes correspondant à la plus petite valeur lisible au
compteur de puissance, qui valent tous les deux $0.001$ mW. Selon le modèle mathématique de 
l'erreur $\Delta C$ et selon les données expérimentales, il est possible de voir que l'erreur maximale 
est en $\theta= 0$ où $C$ est maximal. Ainsi, en remplaçant par les valeurs numériques obtenues lors 
des expériences, il est possible d'estimer l'erreur sur les ordonnées en posant une borne supérieure 
sur la valeur qu'elle peut prendre :

\begin{align*}
  \text{Deux polariseurs : }\Delta C &\leq 0.005\\ 
  \text{Trois polariseurs : }\Delta C &\leq 0.008
\end{align*}

Les erreurs relatives maximales, étant donné que les coefficients à l'angle $\theta= 0$ sont donc de 
0.5\% et de 0.8\% pour deux et trois polariseurs, ce qui est faible et signe d'un résultat avec une
bonne exactitude.
% Source A : https://chem.libretexts.org/Bookshelves/Analytical_Chemistry/Supplemental_Modules_(Analytical_Chemistry)/Quantifying_Nature/Significant_Digits/Propagation_of_Error

\section{Discussion}


\subsection{Analyse des causes d'erreurs}

\subsection{Discussion sur les modèles}
\subsubsection{Analyse des causes d'erreurs}

 Les erreurs présentées dans la partie 1.1 du rapport sont principalement dues aux imperfections des systèmes physiques utilisés, ce qui a pu être observé lors du laboratoire. Notamment lors du calcul de l'intensité après avoir placé le premier polarisateur qui aurait été, dans un cas idéal, divisée par deux peu importe la direction du polarisateur. Les valeurs obtenues dans ce cas sont plutôt de l'ordre de 0.32 fois la valeur initiale. Cette différence s'explique principalement grâce aux imperfections des filtres polarisateurs utilisés, 0.32 est une valeur approximative, variant elle-même d'un filtre à l'autre. Elle doit aussi être due à la lumière présente dans la pièce, dont l'impact parait néanmoins négligeable aux vues des valeurs de minimum mesurées qui avoisinaient les 0.001 mW. De plus, le réseau lampe/lentille positionné en entrée du montage devait produire, dans un cas idéal, un faisceau collimaté jusqu'à la seconde lentille. Lors de la mise en pratique ce faisceau avait une forme conique principalement due aux imperfections de la lampe utilisée dont les rayons divergeaient.

\subsubsection{Lien avec le formalisme de Jones}

En optique, la méthode des matrices permet de décrire l'impact d'un système physique composé de : lentilles, miroirs, filtre, polarisateurs, etc. Sur un faisceau le traversant, en fonction de sa position vis-à-vis de l'axe optique, et de son angle d'entrée. Sorte de fonction de transfert s'exprimant : 

\begin{equation*}
    y_{out} = Ay_{in}
\end{equation*}

Dans le formalisme optique A prend la forme d'une matrice et y d'un vecteur, exactement comme en mécanique quantique. Si on considère $\ket{y}$ comme étant l'état du faisceaux, la fonction de transfert A du système optique peut être interprétée comme un opérateur, tel que : 

\begin{equation*}
    \ket{y_{out}} = \hat{A}\ket{y_{in}}
\end{equation*}

Ainsi pour une polarisation planaire, le modèle de Jones considère une base $\{$$\ket{\hat{x}}$, $\ket{\hat{y}}$$\}$ tel que tout état de polarisation puisse être exprimé comme :

\begin{equation*}
    \ket{y} = a\ket{\hat{x}} + b\ket{\hat{y}} = a\Coord{1}{0} + b\Coord{0}{1}
\end{equation*}

Où : $a^2 + b^2 = 1$ et et $P_2$  sont deux vecteurs orthonormaux. Les photons étant des particules de spin 1 pour lesquels l'état 0 est inaccessible, la base géométrique : $\{$$\ket{\hat{x}}$, $\ket{\hat{y}}$$\}$, fait un lien direct avec la notion de spin quantique au niveau de son formalisme. Le détail de ce formalisme est présenté dans le procédurier du laboratoire. Pour le système de polaristeur utilisé lors du laboratoire on pourrait par exemple avoir : 

\begin{equation*}
    \hat{A} = \hat{P_3}\hat{P_2}\hat{P_1} = \frac{1}{2}\Coord{1 \quad \pm 1}{\pm 1 \quad 1}\Coord{0 \quad 0}{0 \quad 1}\Coord{1 \quad 0}{0 \quad 0}
\end{equation*}

Ou $P_1$ serait alors polarisé horizontalement, $P_2$ verticalement et $P_3$ à $\pm$45°. Cette équation explicite le lien entre le formalisme du spin et celui de la polarisation de la lumière. Sont résultat est bien 0 car, une lumière polarisée horizontalement se retrouve entièrement bloquée par un polarisateur vertical, comme si les états géométriques $\{$$\ket{\hat{x}}$, $\ket{\hat{y}}$$\}$ était des états de spin up and down, comme présenté précédemment (photon : spin 1 dont l'était 0 est inaccessible).

\subsection{Analyse des causes d'erreurs}

\subsection{Discussion sur les modèles}

\subsection{Question 1}
Pour fabriquer un filtre à transmission ajustable, il est possible d'utiliser le même principe que celui réalisé dans cette expérience. En d'autres termnes, en utilisant deux polariseurs linéaires, placés en série avec un angle variable entre leurs axes de polarisation, la transmission devient dépendante de la valeur de l'angle. Ainsi, un filtre de transmission ajustable est créé. Cette méthode fonctionne pour tous les types de faisceaux. Cependant, selon le type de celui-ci, un différent nombre de filtre est nécessaire. Si le faisceau n'est pas polarisé, deux filtres doivent être utilisés alors que, si le faisceau est polarisé, un seul filtre est nécessaire.

\subsection{Question 2}
Pour provoquer une rotation de l'état de polarisation de 90$^\circ$, il est possible d'utiliser une série de polariseurs linéaires orientés à des angles intermédiaires afin de réaliser une rotation progressive. Par exemple, avec trois polariseurs, le premier est placé à un angle de 0$^\circ$, le deuxième à 45$^\circ$, et le troisième à 90$^\circ$. Ce type de dispositif entraîne une perte d'intensité en fonction de l'angle entre les axes de polarisation des polariseurs successifs suivant la loi suivante \textcolor{red}{(Procédurier)}.
\begin{equation}
  I=I_{0}\cos^{2}(\theta)
\end{equation}
De cette manière, pour un dispositif contenant quatre polariseurs, l'efficacité de celui-ci est défini par la valeur du cosinus, soit de $\cos^{6}(\theta)$. L'effet du nombre de polariseurs permet d'augmenter l'efficacité des polariseurs. En effet, l'ajout de polariseurs permet de diminuer les angles entre ceux-ci, permettant de minimiser les pertes causées entre les différentes étapes. Ainsi, en augmentant le nombre de polariseurs, la rotation de la polarisation est davantage graduelle, et la transmission augmente. De cette manière, un nombre infini de polariseurs devrait tendre vers une efficacité parfaite.

\subsection{Question 3}

\subsection{Question 4}
Les vecteurs de Jones sont utiles pour représenter les états de polarisation entièrement polarisés puisque ceux-ci décrivent uniquement l'amplitude et la phase de la polarisation d'une onde lumineuse. Cependant, dans le cas des états partiellement polarisés, un paramètre supplémentaire doit être considéré, soit le degré de polarisation. De cette manière, les paramètres de Strokes sont utilisés pour les états partiellement polarisés puisque cette méthode permet l'algèbre de dimension 4, alors que celle de Jones ne permet qu'un algèbre de dimension 2. En d'autres termnes, les vecteurs de Jones ne permettent pas de décrire les états partiellement polarisés, car ceux-ci n'ont pas toutes les informations nécessaires pour représenter les composantes non polarisés de la lumière, alors que les paramètres de Strokes le permettent.

\section{Conclusion}



\clearpage

% \bibliographystyle{unsrtnat}
% \bibliography{My_Library}

\end{document}
