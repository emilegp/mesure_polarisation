\documentclass[11pt,letterpaper]{article}
\usepackage[top=3cm, bottom=2cm, left=2cm, right=2cm, columnsep=20pt]{geometry}
\usepackage{pdfpages}
\usepackage{graphicx}
\usepackage{etoolbox}
\apptocmd{\sloppy}{\hbadness 10000\relax}{}{}
% \usepackage[numbers]{natbib}
\usepackage[T1]{fontenc}
\usepackage{ragged2e}
\usepackage[french]{babel}
\usepackage{listings}
\usepackage{color}
\usepackage{soul}
\usepackage[utf8]{inputenc}
\usepackage[export]{adjustbox}
\usepackage{caption}
\usepackage{amsmath}
\usepackage{amssymb}
\usepackage{float}
\usepackage{csquotes}
\usepackage{fancyhdr}
\usepackage{wallpaper}
\usepackage{siunitx}
\usepackage[indent]{parskip}
\usepackage{textcomp}
\usepackage{gensymb}
\usepackage{multirow}
\usepackage[hidelinks]{hyperref}
\usepackage{abstract}
\renewcommand{\abstractnamefont}{\normalfont\bfseries}
\renewcommand{\abstracttextfont}{\normalfont\itshape}
\usepackage{titlesec}
\titleformat{\section}{\large\bfseries}{\thesection}{1em}{}
\titleformat{\subsection}{\normalsize\bfseries}{\thesubsection}{1em}{}
\titleformat{\subsubsection}{\normalsize\bfseries}{\thesubsubsection}{1em}{}

\usepackage{xcolor}
\definecolor{codegreen}{rgb}{0,0.6,0}
\definecolor{codegray}{rgb}{0.5,0.5,0.5}
\definecolor{codepurple}{rgb}{0.58,0,0.82}
\definecolor{backcolour}{rgb}{0.95,0.95,0.92}
\lstdefinestyle{mystyle}{
    backgroundcolor=\color{backcolour},   
    commentstyle=\color{codegreen},
    keywordstyle=\color{magenta},
    numberstyle=\tiny\color{codegray},
    stringstyle=\color{codepurple},
    basicstyle=\ttfamily\footnotesize,
    breakatwhitespace=false,         
    breaklines=true,                 
    captionpos=b,                    
    keepspaces=true,                 
    numbers=left,                    
    numbersep=5pt,                  
    showspaces=false,                
    showstringspaces=false,
    showtabs=false,                  
    tabsize=2
}
\lstset{style=mystyle}

\usepackage[most]{tcolorbox}
\newtcolorbox{note}[1][]{
  enhanced jigsaw,
  borderline west={2pt}{0pt}{black},
  sharp corners,
  boxrule=0pt, 
  fonttitle={\large\bfseries},
  coltitle={black},
  title={Note:\ },
  attach title to upper,
  #1
}

%----------------------------------------------------

\setlength{\parindent}{0pt}
\DeclareCaptionLabelFormat{mycaptionlabel}{#1 #2}
\captionsetup[figure]{labelsep=colon}
\captionsetup{labelformat=mycaptionlabel}
\captionsetup[figure]{name={Figure }}
\newcommand{\inlinecode}{\normalfont\texttt}
\usepackage{enumitem}
\setlist[itemize]{label=\textbullet}

\begin{document}
\begin{titlepage}
\center

\begin{figure}
    \ThisULCornerWallPaper{.4}{Polytechnique_signature-RGB-gauche_FR.png}
\end{figure}
\vspace*{2 cm}

\textsc{\Large \textbf{PHS2223 --} Introduction à l'optique moderne}\\[0.5cm]
\large{\textbf{Équipe : 04}}\\[1.5cm]

\rule{\linewidth}{0.5mm} \\[0.5cm]
\Large{\textbf{Expérience 3}} \\[0.2cm]
\text{Mesure de polarisation}\\
\rule{\linewidth}{0.2mm} \\[2.3cm]

\large{\textbf{Présenté à}\\
  Guillaume Sheehy\\
  Esmat Zamani\\[2.5cm]
  \textbf{Par :}\\
  Émile \textbf{Guertin-Picard} (2208363)\\
  Laura-Li \textbf{Gilbert} (2204234)\\
  Tom \textbf{Dessauvages} (2133573)\\[3cm]}

\large{\today\\
Département de Génie Physique\\
Polytechnique Montréal\\}

\end{titlepage}

%----------------------------------------------------

\tableofcontents
\pagenumbering{roman}
\newpage

\pagestyle{fancy}
\setlength{\headheight}{14pt}
\renewcommand{\headrulewidth}{0pt}
\fancyfoot[R]{\thepage}

\pagestyle{fancy}
\fancyhf{}
\renewcommand{\headrulewidth}{1pt}
\fancyhead[L]{\textbf{PHS2223}}
\fancyhead[C]{Mesure de polarisation}
\fancyhead[R]{\today}
\fancyfoot[R]{\thepage}

\pagenumbering{arabic}
\setcounter{page}{1}

%----------------------------------------------------

\section{Résultats}

Suite à la prise de données au laboratoire, il est possible de comparer les valeurs de
coefficients de transmission obtenus aux hypothèses émises pour les mesures à deux et trois polariseurs.
En reprenant la courbe proportionnelle à $\cos^{2}\left( \theta \right)$ pour le montage de
deux polariseurs, la figure \ref{2pol} montre les valeurs obtenues, qui suivent très bien la
tendance prédite.

\begin{figure}[H]
  \centering
  \includegraphics[scale=0.7]{viz_deux_pol.png}
  \caption{Résultats de la prise de mesure superposée à la prédiction de l'hypothèse pour la 
  mesure à deux polariseurs.}
  \label{2pol}
\end{figure}

De même pour le montage à trois polariseurs, la figure \ref{3pol} montre les valeurs de coefficient
de transmission expérimentales, qui suivent aussi bien la courbe proportionnelle à 
$\cos^{4}\left( \theta \right)$ émise par l'hypothèse.

\begin{figure}[H]
  \centering
  \includegraphics[scale=0.7]{viz_trois_pol.png}
  \caption{Résultats de la prise de mesure superposée à la prédiction de l'hypothèse pour la 
  mesure à trois polariseurs.}
  \label{3pol}
\end{figure}

\subsection{Estimation des erreurs}

Les incertitudes présentées avec les barres d'erreurs dans les graphiques ont été trouvées comme suit.
Pour les incertitudes sur l'angle $\theta$, la valeur est tout simplement la moitié de la plus petite
graduation sur les cadrans gradués, multiplié par le nombre de cadrans utilisés. Ainsi, pour les
mesures à deux polariseurs, comme la plus petite graduation était de $1\degree$, l'incertitude 
$\Delta\theta$  sur l'angle est aussi de $1\degree$. Pour trois polariseurs, $\Delta\theta = 1.5\degree$. 

Pour les incertitudes sur les coefficients de transmission $C$, dont le calcul se fait par

\begin{equation}
  C = \frac{P}{P_{0}},
\end{equation}

où $P$ est la puissance mesurée à un certain angle $\theta$ et $P_{0}$ est $P\left( \theta= 0 \right)$.
La propagation d'erreur pour une division de la sorte permet de calculer l'incertitude sur $C$ pour un
certain $P$ \textcolor{red}{source A} :

\begin{equation}
  \Delta C = C\sqrt{\left( \frac{\Delta P}{P} \right)^{2} + \left( \frac{\Delta P_{0}}{P_{0}} \right)^{2}},
\end{equation}

où $\Delta P$ et $\Delta P_{0}$ sont des incertitudes correspondant à la plus petite valeur lisible au
compteur de puissance, qui valent tous les deux $0.001$ mW. Selon le modèle mathématique de 
l'erreur $\Delta C$ et selon les données expérimentales, il est possible de voir que l'erreur maximale 
est en $\theta= 0$ où $C$ est maximal. Ainsi, en remplaçant par les valeurs numériques obtenues lors 
des expériences, il est possible d'estimer l'erreur sur les ordonnées en posant une borne supérieure 
sur la valeur qu'elle peut prendre :

\begin{align*}
  \text{Deux polariseurs : }\Delta C &\leq 0.005\\ 
  \text{Trois polariseurs : }\Delta C &\leq 0.008
\end{align*}

Les erreurs relatives maximales, étant donné que les coefficients à l'angle $\theta= 0$ sont donc de 
0.5\% et de 0.8\% pour deux et trois polariseurs, ce qui est faible et signe d'un résultat avec une
bonne exactitude.
% Source A : https://chem.libretexts.org/Bookshelves/Analytical_Chemistry/Supplemental_Modules_(Analytical_Chemistry)/Quantifying_Nature/Significant_Digits/Propagation_of_Error

\section{Discussion}
Cette section présente l'analyse, l'interprétation et la critique des résultats obtenus de manière expérimentale.

\subsection{Analyse des causes d'erreurs}

\subsection{Discussion sur les modèles}

\subsection{Question 1}
Pour fabriquer un filtre à transmission ajustable, il est possible d'utiliser le même principe que celui réalisé dans cette expérience. En d'autres termnes, en utilisant deux polariseurs linéaires, placés en série avec un angle variable entre leurs axes de polarisation, la transmission devient dépendante de la valeur de l'angle. Ainsi, un filtre de transmission ajustable est créé. Cette méthode fonctionne pour tous les types de faisceaux. Cependant, selon le type de celui-ci, un différent nombre de filtre est nécessaire. Si le faisceau n'est pas polarisé, deux filtres doivent être utilisés alors que, si le faisceau est polarisé, un seul filtre est nécessaire.

\subsection{Question 2}
Pour provoquer une rotation de l'état de polarisation de 90$^\circ$, il est possible d'utiliser une série de polariseurs linéaires orientés à des angles intermédiaires afin de réaliser une rotation progressive. Par exemple, avec trois polariseurs, le premier est placé à un angle de 0$^\circ$, le deuxième à 45$^\circ$, et le troisième à 90$^\circ$. Ce type de dispositif entraîne une perte d'intensité en fonction de l'angle entre les axes de polarisation des polariseurs successifs suivant la loi suivante \textcolor{red}{(Procédurier)}.
\begin{equation}
  I=I_{0}\cos^{2}(\theta)
\end{equation}
De cette manière, pour un dispositif contenant quatre polariseurs, l'efficacité de celui-ci est défini par la valeur du cosinus, soit de $\cos^{6}(\theta)$. L'effet du nombre de polariseurs permet d'augmenter l'efficacité des polariseurs. En effet, l'ajout de polariseurs permet de diminuer les angles entre ceux-ci, permettant de minimiser les pertes causées entre les différentes étapes. Ainsi, en augmentant le nombre de polariseurs, la rotation de la polarisation est davantage graduelle, et la transmission augmente. De cette manière, un nombre infini de polariseurs devrait tendre vers une efficacité parfaite.

\subsection{Question 3}

\subsection{Question 4}
Les vecteurs de Jones sont utiles pour représenter les états de polarisation entièrement polarisés puisque ceux-ci décrivent uniquement l'amplitude et la phase de la polarisation d'une onde lumineuse. Cependant, dans le cas des états partiellement polarisés, un paramètre supplémentaire doit être considéré, soit le degré de polarisation. De cette manière, les paramètres de Strokes sont utilisés pour les états partiellement polarisés puisque cette méthode permet l'algèbre de dimension 4, alors que celle de Jones ne permet qu'un algèbre de dimension 2. En d'autres termnes, les vecteurs de Jones ne permettent pas de décrire les états partiellement polarisés, car ceux-ci n'ont pas toutes les informations nécessaires pour représenter les composantes non polarisés de la lumière, alors que les paramètres de Strokes le permettent.

\section{Conclusion}



\clearpage

% \bibliographystyle{unsrtnat}
% \bibliography{My_Library}

\end{document}
